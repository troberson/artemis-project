\documentclass[11pt,letterpaper,conference]{IEEEtran}

\listfiles
\usepackage{fontspec}
% \usepackage{amsmath}
% \usepackage[mathrm=sym]{unicode-math}
\usepackage[mathrm=sym, warnings-off={
    mathtools-colon,mathtools-overbracket}]{unicode-math}
\setmainfont{Arial}
\setsansfont{Arial}
\setmonofont{MonoLisa}
\setmathfont{FiraMath-Regular.otf}


\input{_preamble/preamble_letter.tex}
\input{_preamble/preamble_float.tex}
\input{_preamble/preamble_units.tex}
\input{_preamble/preamble_table.tex}

\usepackage[export]{adjustbox}

\usepackage{caption}
\usepackage{balance}

\usepackage[backend=biber,      % replace bibtex with biber (bibliography backend engine)
    bibstyle=ieee,              % write literature lists in IEEE style
    citestyle=numeric-comp,     % \cite uses a numeric key
    sortcites=true,
    maxbibnames=3
]{biblatex}
\addbibresource{references.bib}

\usepackage{listings}
\lstset{
    basicstyle = \ttfamily,
    xleftmargin = 1cm,
    frame = single,
    framesep = 1cm,
    numbers = left,
    numbersep = 1.25cm,
    breaklines = true,
    upquote = true
}
\usepackage{titling}
\usepackage{subfiles}

\setlength\parindent{0.25in}

%% DOCUMENT %%%
\begin{document}
\begin{titlepage}
    \begin{minipage}[t]{0.45\textwidth}
        \begin{flushleft}
            \includegraphics[valign=t,width=\textwidth]{images/wsu.png}
        \end{flushleft}
    \end{minipage}
    \begin{minipage}[t]{0.45\textwidth}
        \begin{flushright}
            \includegraphics[valign=t,width=0.3\textwidth]{images/2022.png}
        \end{flushright}
    \end{minipage}


    \centering
    \vspace*{5cm}

    Washington State University \\
    School of Electrical Engineering and Computer Science \\
    EE 416: Electrical Engineering Design \\
    Dr. Jacob Murray

    \vspace*{2cm}
    Artemis Project \\
    Midterm Progress Report of Beta Prototype \\
    February 25, 2022

    \vspace*{1cm}
    Boris Gindlin \\
    Steven Fordham \\
    Tamara Roberson

    \vspace*{1cm}
    Client: Everett Wind Energy Team \\
    Industry Mentor: Dr. Gordon Taub \\
    Faculty Advisor: Dr. Shuzheng Xie
\end{titlepage}

\onecolumn
\tableofcontents
\thispagestyle{plain}
\pagestyle{plain}
\clearpage

\twocolumn
\section{Abstract}

\emph{[TODO]}

\section{Executive Summary}

\emph{[TODO]}

\section{Summary of Business Analysis}

\emph{[Tamara]}

\section{Broader Impacts and Contemporary Issues}

\emph{[Use old material and adjust]}

\section{Results to Date}

Since the demonstration of the alpha prototype, Artemis Project has consolidated
the wiring from the test bench setup. The wind vane, which will be the input
for the yaw control system, was installed on the top of the nacelle. Three
buck converters were installed on one side of the nacelle, and the Ardino
control unit was placed on the other side. The \qty 5V power for the Arduino
and the ground control was installed on the underside of the nacelle. Two
Anderson power pole connectors for the generated power were connected to the
outputs from the buck converter.

After the conclusion of last semester, the first task that was done was
consolidating the wiring from the test bench setup. The wind vane was placed
on the top of the nacelle, all three buck converters were placed on one side
of the nacelle and the Arduino was placed on the other side, with \qty{5}{\V}
and ground busses underneath the nacelle. Two power pole connectors for the
output power connected to the buck convertor outputs. These are shown in
\ref{img:prototype}.

\begin{figure}[th]
    \centering
    \includegraphics[width=0.45\textwidth]{images/prototype.png}
    \caption{Beta wind turbine prototype. (1) Arduino, (2) port-side buck
        converters, (3) wind vane, (4) power pole connectors, (5) ground busses,
        (6) starboard-side buck converters, and (7) rotor axle.}
    \label{img:prototype}
\end{figure}

Two of the buck converters were damaged from increased voltage during a high
speed rotor test and were replaced. However, this test confirmed that the
remainder of the system withstood speeds of over \qty{2000}{RPM}.

The computer functionality for the yaw control has been completed and tested.
The turbine blades must be perpendicular to the wind direction to generate the
maximum power. The wind tunnel blows in a single direction but by rotating the
turbine, winds from different directions can be simulated. If the wind is not
blowing from directly in front of the turbine, the turbine will rotate itself
using a continuous servo so as to face the wind. A digital threshold creates a
hysteresis effect to avoid constant adjustments.

In high winds, the yaw control will be used to rotate the turbine to be parallel
with the wind so as to avoid damage to the turbine blades and potentially
harmful excess power generation, which can cause overheating and fire damage.

\section{Analysis, Modeling, and Simulation Results}

\emph{[TODO]}

\section{Beta Prototype Test Results}
\subsection{Yaw Control}
\label{sec:yaw_control}

For the alpha prototype, the yaw control utilized a 360 degree servo motor to
adjust the direction the turbine faces based on the wind vane. The servo was
not continuous. Once it reached its maximum rotation, it had to rotate the
opposite direction. This caused issues if the wind was coming from near the
servo's maximum or minimum rotational ability. Also, this servo required a
1:1 gear ratio to the nacelle, which required a great deal of torque to turn.
The servo utilized was also large and drew excessive power, making it unsuitable
for lower wind speeds.

The new replacement yaw control servo is continuous and draws much less power.
The limit on rotation is based on the physical wire connection. The new servo,
however produces less torque and is unable to rotate the nacelle. Even with
the addition of 3D-printed gearing and lubrication, the new servo is still
unsuitable because it cannot overcome the static friction on the support.

One possible solution is to add bearings to reduce the friction between the
nacelle and the support. Another solution is to use two of the smaller servos.


\subsection{Load Control}
\label{sec:load_control}

The load control simulation setup consists of a manually-adjustable
\qty{200}{\ohm} rheostat, shown in Figure \ref{img:rheostat}; a Festo four
quadrant dynamometer, with 2:1 belt transmission pulley ratio coupling
mechanism, shown in Figure \ref{img:dyno}; and the beta prototype shown in
Figure \ref{img:prototype}.

The beta prototype rotor axle is coupled to the dynamometer transmission
axle using a coupling adapter with slip and vibration compensation
capability. The dynamometer is controlled by the LabVolt Data Acquisition
and Control Interface installed on the Windows 10 laptop. The power output
of the system is assessed using current measuring functionality of an
onboard Precision Digital Current and Power Monitor - INA 260, positioned
according to number 5 location in Figure \ref{img:prototype}.

The accuracy of the measured current data is verified through the stand-alone
Fluke multimeter. The voltage is measured separately by another independent
multimeter.

\begin{figure}[th]
    \centering
    \includegraphics[width=0.2\textwidth]{images/rheostat.png}
    \caption{\qty{200}{\ohm} \qty{0.7}{\A} rheostat}
    \label{img:rheostat}
\end{figure}
\begin{figure}[th]
    \centering
    \includegraphics[width=0.45\textwidth]{images/dyno.png}
    \caption{Dynamometer with 2:1 coupling belt transmission ratio mechanism}
    \label{img:dyno}
\end{figure}

During the preparation stage of the analysis, the team determined that the
most useful input parameter would be the torque that the prototype's rotor axle
(shown in Figure \ref{img:prototype}, callout 7) experienced at various wind
speeds. Other input parameters are expressed in terms of this torque. The
mechanical engineering subteam of the Everett Wind Energy Team (EWET)
estimated the torque values, utilizing the fundamental relationships described
in Equation \eqref{eq:torque}:

\begin{equation}
    \label{eq:torque}
    \tau_R = \frac{P_w \times c_p}{\omega_\text{rotor}}
    = \frac{c_p(\lambda, \theta) \times \rho \pi R^3 V_w^2}{2\lambda}
    \text{\,,}
\end{equation}
where $\tau(R)$ is the torque experienced by the prototype rotor axle,
$P(w)$ is the power available in the wind, $C(p)$ is the power-degrading
coefficient expressed as a function of the tip speed ratio and pitch angle,
and $\omega(\text{rotor})$ is the rotational speed of the prototype axle in
(\unit{\radian\per\sec}). For the purpose of the analysis, $C(p)$ was taken to
be \num{0.3}, which indicates that only \qty{30}{\percent} of the power
available in the wind would be extractable. The torque values were assumed to
be between \qty{.1}{\newton\m} and \qty{.3}{\newton\m}. This assumption was
based on previous experiments which demonstrated the approximate range of the
available rotor axle speeds relative to the wind speeds.

Thus torque was taken to be the input parameter and generated power is the
output parameter. The load control constants were determined by observing the
Arduino controller operational threshold as one limiting factor and maximum
available power as the other limiting factor. It was determined experimentally
that no more than $\sim$\qty{0.3}{\W} of power was available at
\qty{.1}{\newton\m} constant torque while the Arduino controller was operating.

This result indicates that there is \qty{60}{\mA} of reserve current available
at this torque level for pitch control and data analysis, assuming a
\qty{5}{\V} Arduino supply. At \qty{.3}{\newton\m} torque power output was
measured to be $\sim$\qty{50}{\W}. In this experiment, the rotor's speed was
not captured because it was not a function of pitch control and tip speed ratio
which are not available in the dynamometer simulation setup as controlled
parameters.

\subsection{Pitch Control}

During the alpha prototype development, the turbine was tested in the wind
tunnel by manually adjusting the pitch angle of the blades. The team discovered
that a steep angle of attack produced high torque and low rotational speed.
Conversely, a shallow angle of attack produced low torque and higher rotational
speed. The pitch control was designed to be able to control the torque and
rotational speed of the rotor to extract the maximum power from the wind.

To achieve this, the pitch control mechanism uses a rod which traverses the
main shaft of the rotor. This will move linearly to adjust the pitch to a
steeper or shallower angle of attack. The particulars of the mechanism are
still being designed. However, the team has decided to utilize a 180 degree
servo to control the pitch angle. Mathematically, the rotational motion of the
servo and the linear motion of the shaft are related by Equation
\eqref{eq:shaft}:

\begin{equation}
    \label{eq:shaft}
    \cos\theta = \text{shaft travel} = \frac{Y}{\tan\theta}
    \text{\,,}
\end{equation}
where the distance $Y$ is the fixed distance of the servo center of rotation to
the shaft and $\theta$ is the angle of the servo.
Figure \ref{img:pitch_control} illustrates the relationship of the servo arm
to the linear shaft motion.

\begin{figure}[th]
    \centering
    \includegraphics[width=0.45\textwidth]{images/pitch_control.png}
    \caption{Pitch control shaft diagram}
    \label{img:pitch_control}
\end{figure}

An alternative pitch control mechanism is to use a continuous servo in
combination with a rack and pinion to move the shaft. However, the angle of a
continuous servo cannot be determined, only the speed and direction, which
would create an open control loop. This would require that other inputs, such
as wind and rotor speeds, would be measured after the pitch adjustment to
decide if additional adjustments are required. Reading these additional values
will increase the response time. Therefore, the team decided to use a
traditional non-continuous 180 degree servo.

\subsection{MakerPlot}

\emph{[Tamara]}

\subsection{Wireless communication}

\emph{[Tamara]}

\section{Beta Prototype Validation Results}
\subsection{Power Verification Analysis}

The verification of the system's power output is a complex process that
directly depends on the functionality of the system's controls. The current
progress towards verification was described in Section \ref{sec:load_control}:
``Load Control.'' In summary, the current prototype solution has demonstrated
that it will be able to sustain the predicted power output of the system.

The system has been verified to produce \qty{.3}{\W} of power at
\qty{.1}{\newton\m} of torque and \qty{50}{\W} at \qty{.3}{\newton\m}.

\subsection{Power Quality Analysis}

The Collegiate Wind Competition (CWC) will judge the quality of the turbine's
power output based on its stability over a five second period. The electrical
noise produced by power electronics switching to manage the system's output is a
factor that must be minimized.

A Digilent Analog Discovery Studio board and associated WaveForms software was
used as an oscilloscope to measure the prototype's power output quality. The
results are shown in Figure \ref{img:osc_voltage} and \ref{img:osc_ripple} in
Appendix \ref{apx:images}. These figures demonstrate that the voltage
fluctuation is within \qty{10}{\percent} of the measured mean voltage
magnitude, and the ripple is \qty{2}{\percent} of the same relative magnitude.
These findings, however, are not indicative of the pass-fail status of the
measured power output and further clarification is needed of the terms in which
power quality judgments will be passed.

\section{Summary of Work Remaining this Semester}
\subsection{Mechanical}

\emph{[I talk about this in the beta test results section, so this can probably be condensed down to just a sentence or so.]}

As discussed in Section \ref{sec:yaw_control}, the yaw control is currently
being revised by the mechanical engineering subteam at EWET. During alpha
prototype testing, Artemis Project determined that the existing mechanism was
insufficient to allow the current continuous servo motor to turn the nacelle.
As discussed, a bearing mechanism will be implemented to determine if that
will be sufficient to reduce friction. If not, a second servo motor will be
installed.

The pitch control design is nearing completion. The next stage is to replace
the existing setup with the improved design. Tests will be conducted to
determine the effectiveness of the new pitch control mechanism. The final stage
is to create packaging for the load components to ensure that they are
implemented according to the CWC requirements.


\subsection{Control}

\begin{itemize}
    \item Arduinos
    \item Solid state relays
    \item Buck convertors
\end{itemize}

\subsection{Programing}

Developing the pitch control software will largely require the completion and
testing of the physical mechanism to determine the details of what is needed.
Since the 180 degree non-continuous servo motor was chosen to control the
linear motion of the shaft, some preliminary software development can be
completed. The geometry and mechanics of the shaft relative to the servo are
known. However, we do not yet know how the connections on the other end of the
shaft will affect the rate of change of the pitch. Once the design has been
solidified and a prototype implemented, the relationship between the motor and
the motion of the shaft can be determined experimentally.

For the power curve performance task of the competition, stable power must be
generated at wind speeds from \qtyrange{5}{11}{\m\per\s}. To improve
stability over a fixed blade design, the system needs a power pitch control
algorithm. Tip speed ratios for each wind speed have been calculated
theoretically to produce the most power possible. The buck converters limit the
voltage and the load controls the current, so producing stable power should not
be difficult.



However, the control system
must determine if the tip speed ratio is too high for a certain load and wind
speed in order to avoid stalling.

Turbine blades have a similar shape as an airplane wing, called an airfoil. As
with an airplane, a turbine can suffer a stall. If the angle of attack is too
steep, the air flow can separate from the blade and cause turbulence, pushing
the air in the boundary layer back towards the front of the wing or blade. For
an airplane, this destroys lift, causing the airplane to fall. For a turbine,
the rotational speed will slow significantly. This phenomenon is illustrated in
Figure \ref{img:stall}.

\begin{figure}[th]
    \centering
    \includegraphics[width=0.3\textwidth]{images/stall.jpg}
    \caption{Airfoil at low angle of attack, high angle of attack, and
        stall\cite{stall}.}
    \label{img:stall}
\end{figure}

It may be wise for the software to position the turbine blade angle for a
more conservative tip speed ratio than calculated, to improve stability at the
cost of some output power. This way the program can have predetermined pitch
angles for each wind speed that can be set and left alone unless prompted
otherwise. Once the desired rotational speed is met then the load will be
increased to the next step.

The durability aspect of the competition requires that the turbine operate at
variable winds speeds from \qtyrange{6}{22}{\m\per\s} for five minutes. The
turbine must be able to produce power across this range. Artemis Project's
primary concern, and the concern in producing an offshore wind farm in
hurricane territory, is to keep the turbine's blades from being damaged or
even torn off in a high wind situation. The pitch control system can adjust
the blade's angle of attack to reduce speed and avoid damage in such cases,
in addition to maximizing power output in normal conditions.

However, even when trying to reduce blade rotational speed to protect the
integrity of the turbine, the system must still be able to maintain a
consistent power output. With the current KV rating of the turbine, this is
about $1400 \pm 300$\;RPM, which will likely be maintained if we can achieve
low enough tip speed ratios. If low tip speed ratios cannot be achieved
at high wind speeds due to mechanical limitations, this will require a blade or
pitch mechanism redesign.


\subsection{Power}

A second Arduino connected to the load will receive commands from the turbine's
Arduino controller and pass them to a bank of solid state relays. The
Arduino will switch between load resistance values using the onboard
controller. The optimal values will be determined experimentally. This will
need to be done after the mechanical subteam has implemented the relay system.

\section{Conclusion}

\emph{[TODO]}
\balance

\raggedright
\printbibliography

\clearpage
\onecolumn
\appendices
\section{Images}
\label{apx:images}

\FloatBarrier
\begin{figure}[th]
    \centering
    \includegraphics[width=\textwidth]{images/osc_voltage.png}
    \caption{Voltage fluctuation}
    \label{img:osc_voltage}
\end{figure}
\begin{figure}[th]
    \centering
    \includegraphics[width=\textwidth]{images/osc_ripple.png}
    \caption{Ripple}
    \label{img:osc_ripple}
\end{figure}

\clearpage
\section{Material Costs}
\label{apx:costs}
\begin{table}[th]
    \centering
    \begin{NiceTabular}{lrl}
        \toprule
        \thead{Product} & \thead{Price (USD)} & \thead{Comment} \\
        \midrule
        \qty{35}{\W} Mini Disc Generator Coreless Generator Three-Phase Permanent &
        \$319.60 & \\
        Rheostat \qty{100}{\W} \qty{100}{\ohm} \qty{1000}{\volt} Std Shaft & \$94.28 & \\
        Rheostat \qty{100}{\W} \qty{200}{\ohm} \qty{1000}{\volt} Std Shaft & \$114.71 & \\
        ALITOVE DC \qty{12}{\V} \qty{5}{\A} Power Supply Adapter Converter Transformer AC & \$11.89 & Unused \\
        MakerPlot licenses & \$236 & \\
        Adafruit 4226 INA260 High or Low Side Voltage, Current, Power Sensor & \$39.80 & Unused (3/4) \\
        Round Tube, 304/304L Stainless Steel, 0.065" Wall Thickness, 5/16" OD & \$13.94 & Research Material \\
        Zener Diodes: \qty{30}{\V} \qty{5}{\W} & \$1.44 & Research Material \\
        Resistors, \qty{25}{\ohm}, \qty{35}{\W}, 5\% & \$6.72 & Research Material \\
        Thick Film Resistors: Through Hole \qty{75}{\ohm} \qty{35}{\W} 5\% TOL & \$6.86 & Research Material \\
        Schottkey Diodes, BOJACK 1N5822, \qty{3}{\A} \qty{40}{\V} DO-201AD & \$5.99 & Replenish WSU stock \\
        MOSFET SMOS Low RON Nch Io: \qty{0.4}{\A} Vdss: \qty{60}{\V} Vgss & \$3.08 & Unused \\
        Potentiometer, Max power \qty{400}{\W} Max R-\qty{200}{\ohm} & \$25.50 & \\
        Mini Electric Linear Actuator Stroke 2 & \$29.99 & Research Material \\
        FEETECH 35KG Continuous Rotation Servo Motor \ang{360} High Torque & \$27.99 & Research Material \\
        DC-DC Buck Boost Converter Module \qtyrange{5.5}{30}{\V} \qty{12}{\V} to \qtyrange{0.5}{30}{\V} & \$25.98 & Research Material \\
        DC\qtyrange{12}{24}{\V} miniature electromagnetic clutch & \$10.25 & Unused \\
        Modern Device Wind Sensor Rev. P & \$24.00 & \\
        DC Buck Converter, DROK DC to DC Step Down Power Supply Module & \$59.98 & \\
        SainSmart 8-Channel \qty{5}{\V} Solid State Relay Module Board for Arduino & \$19.99 & \\
        Misc (connectors, heatshrink tuning, epoxy, PLA str.) & \$89.89 & Manufacturing expense \\
        \midrule
        Total & \$1167.87 \\
        Est. Tax + Shipping (~15\%) &  \$175.18 \\
        Grand Total & \$1343.05 \\
        Total Unused & \$25.21 \\
        Total Research & \$112.92 \\
        \bottomrule
    \end{NiceTabular}
\end{table}

\clearpage
\section{Gantt Chart}
\label{apx:gantt}
\FloatBarrier
\begin{figure}[th]
    \centering
    % \includegraphics[width=\textwidth]{images/gantt-1.png}
\end{figure}
\begin{figure}[th]
    \centering
    % \includegraphics[width=\textwidth]{images/gantt-3.png}
\end{figure}
\begin{figure}[th]
    \centering
    % \includegraphics[width=\textwidth]{images/gantt-4.png}
\end{figure}

\end{document}
