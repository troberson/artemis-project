\documentclass[../ewet_cwc_report.tex]{subfiles}

\begin{document}
\section{Introduction}

\noindent
Our team has picked the horizontal axis turbine because we
wanted a more conservative approach in wing energy generation,
compared to the previous year. The team's main goal this year
was to come up with a functional generation system, that could
create a base line for future experiments and designs. The
advantage of the horizontal axis was in the fact that its
design concept has been well research and a wealth of
information is available on the topic. This way more time
could be spent, trying out different working designs in the
attempt to develop our own. Nevertheless, a fair share of
challenges had to be overcome by the team. One of the
critical concerns this year is involving the weight and
balancing of the prototype since this year's competition has an
offshore theme and the turbine structure will need to reside on
the simulated ocean floor. Because of this reason, vibration,
weight, balancing, and aerodynamics became very important design
criteria. To maintain minimal weight and have adequate
aerodynamical properties, there was no other option but to use
the 3D printing technology. This insured light weight of the
structural elements, fast prototype turnaround and allowed to
model and produce complex aerodynamic shapes of the turbine
housing structure. Another set of technical challenges was due
to the active pitch control design, initial attempts to control
the actual degree of the pitch did not come to fruition,
because the team did not find a solution to accurately control
the position of the blades. This led to a more elegant idea of
the ``speed control'', where turbine control system is
monitoring rate of speed change, and makes decisions to pitch
up, pitch down or to do nothing if it is at the target speed.
Active yaw control was mostly for the team's own research to
develop understanding on how larger wind turbines that do not
have passive yaw control are positioned into the wind. The
original load control idea was to have a continuous load
control, the team has settled for discrete control because
resistive load range was not yet understood in the beginning of
the design process and since the CWC rules wind speeds are also
discreet a simpler wind sensor dependent load control solution
was adopted. Generator selection just happened to naturally
work out in terms of wight distribution, because its stack
could be evenly split over the center of the pole, this has
proven to be one of the benefits of axial flux permanent
magnet (AFPM) machines, compared to permanent magnet
brushless machines that need counter balancing.  An additional
feature of the AFPM was the fact that its fundamental
geometrical design properties provide for relatively large
hollow center in its stator. This gives significant flexibility
to the pitch actuator design approach, which our mechanical
team has successfully implemented using linear actuator
mechanism powered by DC/Stepper motor.

\end{document}
