\documentclass[../ewet_cwc_report.tex]{subfiles}

\begin{document}
\section*{Executive Summary}
\label{sec:summary}

\noindent
This report summarizes collective knowledge and documents to
date practical results in the field of wind power generation,
achieved by WSU-EvCC cross institutional team. Additionally, a
portion of this work concerning electrical engineering, signify
a senior capstone project achievement for the WSU electrical
engineering team. Current academic year's success recognizes
and builds on the previous year's wind energy team's extensive
research and is grateful for their intellectual legacy. This
year, the team has taken a top-down approach in research and
development of the prototype. It was decided to avoid extensive
fundamental research and study of competitors achievements.
Design priority was given to the commercially available
components based on a trial-and-error approach. This permitted
relative freedom from the burdens of predisposition to operate
in the wake of someone else's success, allowing more
experimental courage and satisfaction with the accomplishments.
This year, the team has returned to the traditional
horizontal-axis wind turbine design with autonomous pitch, yaw,
and load control. Although autonomous yaw control was beyond
the CWC requirements, the design experience was determined to
be beneficial to the achievements within the scope of the
senior capstone project, and perhaps future teams' research.
The electrical team has expanded on the previous year's turbine
and load control component ideas and developed its own robust
approach to power management, voltage regulation and generator
selection. The mechanical team had less luck with previous
year's work since very few design solutions of last year's
vertical-axis turbine design were applicable in the
horizontal-axis design. After initial experiments and
conceptual deliberation, the control team has settled with
selecting the rotational speed of the machine as the primary
pitch control input and wind speed as primary load control
input, implementing separate controllers for each device
connected via communication bus. Additionally, beyond the CWC
requirements scope, some team member's time was dedicated to
the development of the HMI, data acquisition and live power
output monitoring systems, this was done with consideration
for broader wind farm project development. For this purpose,
MakerPlot software was chosen, and a suitable application was
developed, however because of limited competence in this field
of work and limited human resources it was not integrated into
the final design. The immediate state of the prototype and the
project progression is determined satisfactory. The team was
able to achieve its selected objectives in control of the
turbine and power generation. Work is continuing to finalize
turbine-load communication, final blade, and foundation design
along with revisions to the pitch actuator mechanism.

\end{document}
