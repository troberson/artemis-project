\documentclass[11pt,letterpaper]{article}

\input{_preamble/preamble_letter.tex}
\input{_preamble/preamble_left}
\input{_preamble/preamble_table.tex}
\input{_preamble/preamble_titlepage.tex}

\setcounter{tocdepth}{2}

\usepackage{titletoc}
\contentsmargin{2.55em}
\dottedcontents{section}[4em]{\fillast}{1.5em}{1em}
\dottedcontents{subsection}[6em]{\fillast}{2em}{1em}

\usepackage{fancyhdr}
\usepackage{lastpage}
\pagestyle{fancy}
\setlength{\headheight}{16pt}
\fancyhead[L,C]{}
\fancyhead[R]{Artemis Project Charter}
\fancyfoot[L,C]{}
\fancyfoot[R]{{\thepage} of \pageref{LastPage}}
\renewcommand{\headrulewidth}{0.5pt}
\renewcommand{\footrulewidth}{0.5pt}

%% DOCUMENT %%%
\begin{document}

\title{Artemis Project Charter}
\author{Tamara Roberson}

\begin{center}
    \vspace*{5cm}

    {\Huge Artemis Project \\
        Charter}

    \vspace*{1cm}

    Washington State University \\
    School of Electrical Engineering and Computer Science

    \vspace*{1cm}

    2022-01-15

    \vspace*{1cm}

    Prepared by: \\
    Tamara Roberson
    \vspace*{1cm}

    Authorized by: \\
    Dr. Gordon Taub
\end{center}
\clearpage

\tableofcontents

\clearpage

\section{Executive Summary}
This document provides the Initiation stage description of the Artemis
Project. It authorizes the Project Manager to prepare a complete and accurate
project plan for presentation to the Sponsor for final decision making and
potential approval for proceeding to the Execution stage.

This project provides the essential electrical and electronic components for
the development of a wind turbine prototype to be entered into the US Department
of Energy Collegiate Wind Competition.

If the project plan is approved, an updated Execution stage charter shall be released authorizing release of funds and project execution start.

\section{Sponsor}
Dr. Gordon Taub, faculty advisor for the Everett Wind Energy Team (EWET), is
the sponsor of this project, responsible for the project funding, and the
authority for approval of the project plan and any changes to the baseline if
approved for execution.

\section{Customer}
Kaitlin Jones, president of the Everett Wind Energy Team, is the customer of
the project, representing the end-users of the project, and responsible for participation in definition of the project scope, and responsible for sign-off of the requirements and acceptance of the final deliverables if the project is approved for proceeding to execution.

\section{Scope}
This section describes the project objective and current assumptions and constraints.

\subsection{Objective}
The project objective is to develop the electronic and electrical components
of a wind turbine prototype to be entered into the US Department of Energy
Collegiate Wind Competition.

\subsection{Assumptions}
The project assumptions are:

[High-level items that planning should assume are true.]

\subsection{Constraints}
The project constraints are:

[High-level items that limit planning, typically including any budget or schedule targets, high-level standards or site limitation, etc.  Leave as much to the requirements as possible.]

\section{Conceptual Solution}
The conceptual solution to meet the project objective is ...

    [Top-level description of the project result, ideally with a very top-level diagram if possible.  Sometimes this is just a very top-level list of the Major Deliverables.  This conceptual solution may have been selected from more than one option considered during the business case analysis.]

\section{Business Case}
The initial business case analysis, to be reassessed on finalization of planning, currently indicates the project benefit is greater than the estimate cost, as described below ...

    [Estimate of benefits and costs, return on investment, or other measure, to the organization's standard on accuracy at Initiation, typically somewhere anywhere +/- 50\% and -100\%/+200\%, often with reference to a separate business case document for additional details.]

\subsection{Options}
The following options were examined:

[Brief listing of the options considered, and reasons for rejection or selection, often including the three options (a) status quo rejected since the costs or lost opportunity are too high, (b) a sensible best value option with reasonable costs and good benefits, (c) a high cost option showing it was too expensive and making the point that the second best value option is a carefully considered best choice. ]

\subsection{Assumptions}
The business case is based on the following assumptions:

[Assumptions and references to supporting documents on which the business case depends.]

\subsection{Benefits}
The benefits are expected to be as follows ...

    [Increase in revenue, efficiency, productivity, decrease or avoidance of costs.]

\subsection{Costs}
The costs are estimated to be as follows ...

    [By a comparison to a similar project, or a simple and top-level roll-up estimate of the people costs, material, and services.]

\subsection{Analysis}
The comparison of benefits to costs shows the selected option has a positive [Benefit / Cost Ratio (BCR) or Return On Investment (ROI) or net present value (NPV) with year as applicable].

\section{Stakeholders}
This section describes the [ABC] key stakeholders.  The project stakeholders are affected and can affect the project, and therefore will be included in definition of the project scope and development of the project plan, and will be included in regular communications as the project progresses if approved for execution.

The key project stakeholders, along with their role, key need, priorities, and planned communications are described in the Stakeholder Register provided below.

    [Name, role / title, key need, priorities (scope, schedule, budget), and planned communications.]

\section{Issues}
Significant issues identified at the initiation stage to be considered during planning include ...

    [Items that require further analysis and could significantly affect scope, time, or cost.]

\section{Risks}
Significant risks identified at the initiation stage to be further considered during planning are described below.

    [Risks that require further analysis and could significantly affect scope, time, or cost, and are uncertain wrt probability at this stage.]

\section{Project Manager}
The assigned Project Manager shall prepare an optimized plan documenting the project scope, schedule, budget, and risks for sponsor review before proceeding to the execution stage.

The Project Manager is authorized to call on support from within and outside the organization as required to prepare a complete and accurate plan.

The Project Manager, planning budget, and planning schedule are defined as follows:

• Project Manager:  [Name, contact information.]

• Budget:  [Typically 2.5\% to 5\% of the business case analysis guesstimate of cost or hours.]

• Schedule:  [Expected date for review of the final plan, and any other interim reviews.]

\end{document}
